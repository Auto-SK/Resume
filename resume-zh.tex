%%
%% Copyright (c) 2018-2019 Weitian LI <wt@liwt.net>
%% CC BY 4.0 License
%%
%% Created: 2018-04-11
%%

% Chinese version
\documentclass[zh]{resume}

% Adjust icon size (default: same size as the text)
% \iconsize{\Large}

% File information shown at the footer of the last page
\fileinfo{%
  \faCopyright{} 2020--2021, SUN Kai \hspace{0.5em}
  \creativecommons{by}{4.0} \hspace{0.5em}
  \githublink{Auto-SK}{Resume} \hspace{0.5em}
  \faEdit{} \today
}

\name{凯}{孙}

% \keywords{BSD, Linux, Programming, Python, C, Shell, DevOps, SysAdmin}

% \tagline{\icon{\faBinoculars}} <position-to-look-for>}
% \tagline{<current-position>}

\photo{2.5cm}{pics/avatar.jpg}

\profile{
  \mobile{(+86) 18616366601}
  \email{sunkai@codingcat.cc}
  \iconlink{\faLink}{https://codingcat.cc}{https://codingcat.cc} \\
  \university{上海科技大学}
% \degree{电子科学与技术 \textbullet 硕士}
  \birthday{1996-03-23}
  \address{上海 \textbullet 浦东新区}
  \github{Auto-SK}
  % Custom information:
  % \icontext{<icon>}{<text>}
  % \iconlink{<icon>}{<link>}{<text>}
}

\begin{document}
\makeheader

%======================================================================
% Summary & Objectives
%======================================================================
{\onehalfspacing\hspace{2em}%
    2022届硕士研究生,期望\textbf{软件开发}实习岗位。熟悉 Python,了解 C/C++/C\#、Java 等编程语言,具有良好的编程习惯及文档编写能力;会使用 Git、SSH、Docker 等开发工具,了解 SQL (CRUD) ;使用过 Pycharm、IntelliJ IDEA、Visual Studio 等开发环境;会基本的 Linux 操作,有 VPS 使用和个人网站部署经历;擅长自主学习和自主解决问题,能够高效地进行团队沟通和协作。
    \par}

%======================================================================
\sectionTitle{教育背景}{\faGraduationCap}
%======================================================================
\begin{educations}
    \education%
    {2019.08}%
    {上海科技大学}%
    {信息科学与技术学院}%
    {电子科学与技术}%
    {硕士}

    \separator{0.5ex}
    \education%
    {2015.08}%
    [2019.06]%
    {西安理工大学}%
    {自动化与信息工程学院}%
    {自动化}%
    {学士}
\end{educations}

%======================================================================
\sectionTitle{技能清单}{\faWrench}
%======================================================================
\begin{competences}
    %    \comptence{专业技能}{
    %        软件开发,建模仿真,硬件电路设计,FPGA 开发
    %    }
    \comptence{\icon{\faCode} 编程}{%
        \textbf{Python}, MATLAB, \textbf{C}/C++/C\#, Java, Verilog HDL, \LaTeX
    }
    \comptence{\icon{\faTools} 工具}{%
        \textbf{Git}, SSH, Docker, SQL (CRUD), Markdown, Anaconda, PyQt
    }
    \comptence{开发环境}{%
        \textbf{PyCharm}, Visual Studio, \textbf{Visual Studio Code}, IntelliJ IDEA
    }

    %    \comptence{建模仿真}{%
    %        数值计算,有限元分析
    %    }
    \comptence{操作系统}{%
        \icon{\faWindows} Windows,
        \icon{\faLinux} Linux
        %    		\icon{\faApple} Mac
    }
    \comptence{\icon{\faLanguage} 语言}{
        英语四级:598,英语六级:443
    }
\end{competences}

%======================================================================
\sectionTitle{竞赛获奖}{\faTrophy}
%======================================================================
\begin{awards}
    \award
    {国际级}%
    {
        \begin{itemize}
            \item \datedline{2018年美国大学生交叉学科建模竞赛\textbf{二等奖}}{2018 年 04 月}
        \end{itemize}}
    \separator{0.5ex}
    \award
    {国家级}
    {
        \begin{itemize}
            \item \datedline{2017年全国大学生电子设计竞赛\textbf{全国二等奖} ( \textbf{控制组陕西省第一名} ) }{2017 年 12 月}  %  前 \textbf{5.6\%} 
            \item \datedline{2017年全国大学生数学建模竞赛\textbf{全国二等奖} ( 前 \textbf{3.4\%} ) }{2017 年 11 月}
            \item \datedline{2017年全国大学生智能汽车竞赛\textbf{全国二等奖}}{2017 年 08 月}
        \end{itemize}
    }
    \separator{0.5ex}
    \award
    {省部级}
    {
        \begin{itemize}
            \item \datedline{2017年全国大学生电子设计竞赛陕西赛区\textbf{省级一等奖}}{2017 年 10 月}
            \item \datedline{2017年全国大学生智能汽车竞赛西部赛区\textbf{省级一等奖}}{2017 年 07 月}
            \item \datedline{2018年陕西工科五校 ( TI ) 杯校际联赛\textbf{省级二等奖}}{2018 年 06 月}
            \item \datedline{2017年陕西工科五校 ( TI ) 杯校际联赛\textbf{省级三等奖}}{2017 年 06 月}
        \end{itemize}
    }
    %    \separator{0.5ex}
    %    \award
    %    {校级}
    %    {
    %        \begin{itemize}
    %            \item \datedline{西安理工大学2018年大学生电子设计与技能竞赛\textbf{一等奖}}{2018 年 06 月}
    %            \item \datedline{西安理工大学第二十六届大学生课外学术科技竞赛\textbf{特等奖}}{2017 年 12 月}
    %            \item \datedline{西安理工大学第二十六届大学生课外学术科技竞赛\textbf{三等奖}}{2017 年 12 月}
    %            \item \datedline{西安理工大学2017年“互联网+”大学生创新创业大赛\textbf{一等奖}}{2017 年 10 月}
    %                  %		\item \datedline{西安理工大学第二十六届“理奥杯”大学生课外学术科技作品竞赛\textbf{特等奖}}{2017 年 11 月}
    %                  %		\item \datedline{西安理工大学第二十六届“理奥杯”大学生课外学术科技作品竞赛\textbf{一等奖}}{2017 年 11 月}
    %            \item \datedline{西安理工大学第二十五届大学生课外学术科技竞赛\textbf{一等奖}}{2017 年 05 月}
    %                  %		\item \datedline{西安理工大学第二十五届“理奥杯”大学生课外学术科技作品竞赛\textbf{三等奖}}{2016 年 11 月}
    %        \end{itemize}
    %    }

\end{awards}

%======================================================================
\sectionTitle{奖励荣誉}{\faAward}
%======================================================================
\begin{awards}
    \award
    {国家级}
    {
        \begin{itemize}
            \item \datedline{2016-2017年度国家励志奖学金}{2017 年 12 月}
            \item \datedline{2015-2016年度国家励志奖学金}{2016 年 11 月}
        \end{itemize}
    }
    \separator{0.5ex}
    \award
    {校级}
    {
        \begin{itemize}
            \item \datedline{2020年度学业奖学金(¥8000)}{2020 年 10 月}
            \item \datedline{2019年度学业奖学金(¥8000)}{2019 年 10 月}
            \item \datedline{吴江创新奖学金}{2018 年 10 月}
            \item \datedline{创新成果一等奖}{2018 年 06 月}
            \item \datedline{2016-2017年度尚真笃学先进个人}{2017 年 11 月}
            \item \datedline{2015-2016年度三好学生标兵}{2016 年 11 月}
            \item \datedline{2015-2016年度优秀团员}{2016 年 05 月}
        \end{itemize}
    }
\end{awards}

%======================================================================
\sectionTitle{项目经历}{\faListUl}
%======================================================================
\begin{experiences}
    \experience%
    [2020.03]
    {现在}%
    {\textbf{中广核涡流检查数据仿真软件开发}}%
    [%~~~~ {主要负责文献搜索与整理、数据收集与处理、数学模型建立与求解:}
        \begin{itemize}
            \item {数值仿真:编写有限元网格划分程序和有限元求解程序;}
            \item {程序接口:编写计算后端和界面前端接口,传输两端数据;}
            \item {软件测试:参与软件整体测试并修改其中漏洞;}
            \item {文档撰写:编写接口说明文档和软件操作手册;}
            \item {版本控制:搭建并维护软件仓库,协调团队资源。}
        \end{itemize}]
    [Python, PyQt, Matplotlib, MATLAB, C\#, FEM, Git]
    \separator{0.5ex}

    \experience%
    [2019.08]
    {2020.05}%
    {\textbf{中广核无损检测探头阻抗仿真}}%
    [%~~~~ {主要负责文献搜索与整理、数据收集与处理、数学模型建立与求解:}
        \begin{itemize}
            \item {数值计算:编写数值计算程序,计算探头的等效电路参数,进而推导出阻抗;}
            \item {界面显示:使用 PyQt5 编写程序界面,传递探头参数并绘制阻抗特性曲线;}
            \item {软件测试:对软件进行测试并且修改软件漏洞;}
            \item {程序发布:编写软件开发文档、打包程序并发布软件。}
        \end{itemize}]
    [Python, MATLAB, PyQt, Matplotlib, Numpy, FEM, Git]
    \separator{0.5ex}

    \experience%
    [2020.08]
    {现在}%
    {\textbf{个人网站部署与建设}}%
    [%~~~~ {主要负责文献搜索与整理、数据收集与处理、数学模型建立与求解:}
        \begin{itemize}
            \item {网站部署:在腾讯云主机部署网站,并进行 ICP 备案和公安备案;}
            \item {SSL 证书:配置并使用 acme.sh 自动申请、部署 SSL 证书,开启全站 HTTPS ;}
            \item {SEO 优化:部署网站 Robots.txt 和 Sitemap,实现网站百度收录和 Google 收录;}
            \item {域名邮箱:使用 Exchange 部署域名邮箱,实现个人的免费域名邮箱;}
            \item {个人云盘:使用 OneDriveIndex 搭建个人的免费云盘。}
        \end{itemize}]
    [Typecho, PHP, HTML, CSS, Node.js, Shell, OneDriveIndex]
    \separator{0.5ex}

    %    \experience%
    %    [2017.12]%
    %    {2018.02}%
    %    {\textbf{美国大学生交叉学科建模竞赛(E题: 气候变化如何影响地区的不稳定性)}}%
    %    [%~~~~ {主要负责文献搜索与整理、数据收集与处理、数学模型建立与求解:}
    %        \begin{itemize}
    %            \item {文献搜索与整理:在相关数据库搜索文献,阅读文献后对相关文献进行整理;}
    %            \item {数据收集与处理:收集相关数据,利用MATLAB、SPSS等软件进行数据处理;}
    %            \item {数学模型的建立:建立国家脆弱指数模型与多元线性回归模型;}
    %            \item {数学模型的求解:使用SPSS求解上述多元线性回归模型,代入数据验证模型。}
    %        \end{itemize}]
    %    [MATLAB, SPSS, \LaTeX, 多元线性回归]
    %    \separator{0.5ex}

    %    \experience%
    %    [2017.04]%
    %    {2017.09}%
    %    {\textbf{全国大学生数学建模竞赛(B题:“拍照赚钱”的任务定价)}}%
    %    [%~~~~ {主要负责文献搜索与整理、数据收集与处理、数学模型建立与求解、软件算法编写与调试:}
    %        \begin{itemize}
    %            \item {文献搜索与整理:在相关数据库搜索文献,阅读文献后对相关文献进行整理;}
    %            \item {数据收集与处理:绘制热力图和位置分布图,对数据进行聚类分析,得到聚类中心点;}
    %            \item {数学模型建立与求解:构造指数模型,建立并求解广义多元回归模型;}
    %            \item {软件算法编写与调试:根据约束条件利用多种群遗传算法求解出最优化方案。}
    %        \end{itemize}]
    %    [MATLAB, 聚类分析, 热力图, 多元回归, 遗传算法]
    %    \separator{0.5ex}
    %
    %    \experience
    %    [2017.06]
    %    {2017.08}
    %    {\textbf{全国大学生电子设计竞赛(B题:滚球控制系统)}}
    %    [%~~~~ {主要负责PCB设计与焊接、硬件部分设计与搭建、系统参数调节与整定:}
    %        \begin{itemize}
    %            \item {PCB设计与焊接:绘制电赛系统板的原理图和PCB图,焊接并调试;}
    %            \item {硬件部分设计与搭建:根据赛题要求,与队友合作设计并搭建硬件平台;}
    %            \item {系统参数调节与整定:硬件平台搭建完毕后与队友一起调试参数。}
    %        \end{itemize}]
    %    [Altium Designer, SCH, PCB, 焊接, 调试]
    %    \separator{0.5ex}

    \experience
    [2017.02]
    {2017.08}
    {\textbf{全国大学生智能汽车竞赛(光电直立组)}}
    [%~~~~ {主要负责PCB设计与焊接、硬件部分设计与搭建、系统参数调节与整定:}
        \begin{itemize}
            \item {原理图设计:绘制智能车主控部分、驱动部分、电源模块的原理图;}
            \item {PCB设计与焊接:绘制与原理图对应的 PCB 图,焊接电路板并调试;}
            \item {硬件部分搭建:组装车模、调节摄像头和陀螺仪等传感器;}
            \item {系统整体调试:车模硬件搭建完毕后与队友一起调试参数。}
        \end{itemize}]
    [Altium Designer, SCH, PCB, 焊接, 调试]
\end{experiences}

%======================================================================
\sectionTitle{校园经历}{\faUsers}
%======================================================================
\begin{table}[!htbp]
    \begin{tabular}{ll}
        2017年06月 --- 2018年04月 & 西安理工大学大学生科技协会 \textbf{副主席}     \\
        %		2017年06月 --- 2018年06月 & 西安理工大学自动化科学与技术协会技术团成员 \\
        2017年05月 --- 2018年04月 & 西安理工大学工程训练中心创客工坊 \textbf{社长} \\
        2016年06月 --- 2017年06月 & 西安理工大学自动化科学与技术协会 \textbf{干事} \\
        2016年06月 --- 2017年05月 & 西安理工大学工程训练中心创客工坊 \textbf{社员} \\
    \end{tabular}
\end{table}

%======================================================================
\sectionTitle{个人总结}{\faHeart}
%======================================================================
\begin{table}[!htbp]
    \begin{tabular}{rl}
        热爱技术,有很强的自主学习和解决问题能力,能够高效地进行团队沟通和协作。
    \end{tabular}
\end{table}

%%======================================================================
%\sectionTitle{计算机技能}{\faCogs}
%%======================================================================
%\begin{itemize}
%	\item DragonFly BSD 操作系统开发者:
%	200+ 代码提交;内核以及系统工具;
%	在邮件列表和 IRC 频道交流和回答问题
%	\item 使用 Ansible 管理 VPS,部署个人域名邮箱、权威 DNS、网站、Git、IRC 等服务
%	\item 搭建并管理课题组的工作站、计算集群(4 节点)和网络设备
%	\item 参与配置和测试上海天文台的 SKA 高性能计算集群原型机
%	(1 管理节点 + 1 存储节点 + 4 计算节点)
%	\item 设计并开发了\enquote{2014 第一届中国—新西兰联合 SKA 暑期学校}的整个网站
%	(Django, Bootstrap, jQuery)
%\end{itemize}
%
%%======================================================================
%\sectionTitle{个人项目}{\faCode}
%%======================================================================
%\begin{itemize}
%  \item \link{https://github.com/liweitianux/atoolbox}{\texttt{atoolbox}}:
%    (Python, Shell)
%    多年来累积的各种工具,帮助管理系统、执行常用任务、分析天文数据等
%  \item \link{https://github.com/liweitianux/dfly-update}{\texttt{dfly-update}}:
%    (Shell)
%    DragonFly BSD 系统更新程序
%  \item \link{https://github.com/liweitianux/openrcs}{\texttt{openrcs}}:
%    (C)
%    改进 \texttt{OpenBSD RCS},使其与 \texttt{GNU RCS} 足够兼容
%  \item \link{https://github.com/liweitianux/fg21sim}{\texttt{fg21sim}}:
%    (Python)
%    模拟低频射电天空图像
%  \item \link{https://github.com/liweitianux/cdae-eor}{\texttt{cdae-eor}}:
%    (Python, Keras)
%    使用卷积去噪自动编码器(CDAE)分离宇宙再电离(EoR)信号
%  \item \link{https://github.com/liweitianux/chandra-acis-analysis}{\texttt{chandra-acis-analysis}}:
%    (Python, Shell, Tcl)
%    X~射线天文观测数据的半自动化分析程序
%  \item \link{https://github.com/liweitianux/resume}{\texttt{resume}}:
%    (\LaTeX)
%    \emph{此简历}的模板和源文件
%\end{itemize}
%
%%======================================================================
%\sectionTitle{科研成果}{\faAtom}
%%======================================================================
%\begin{itemize}
%  \item 开发低频射电天空图像模拟软件:
%    \link{https://github.com/liweitianux/fg21sim}{\texttt{FG21sim}}
%  \item 开发程序实现~X~射线天文观测数据的半自动化分析:
%    \link{https://github.com/liweitianux/chandra-acis-analysis}{\texttt{chandra-acis-analysis}}
%  \item 利用卷积去噪自动编码器(CDAE)在频率维度分离微弱的宇宙再电离(EoR)信号
%  \item 利用卷积神经网络(CNN)对 FIRST 巡天的射电星系图像根据形态特征进行分类
%  \item 显著改进星系团射电晕的建模,并考虑低频干涉阵列的复杂仪器效应
%  \item 改进~X~射线光谱拟合的背景成分建模,获到更准确可靠的拟合结果
%  \item 发表 2 篇第一作者以及 8 篇合作者 SCI 论文
%\end{itemize}

\end{document}
