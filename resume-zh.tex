%%
%% Copyright (c) 2018-2019 Weitian LI <wt@liwt.net>
%% CC BY 4.0 License
%%
%% Created: 2018-04-11
%%

% Chinese version
\documentclass[zh]{resume}

% Adjust icon size (default: same size as the text)
% \iconsize{\Large}

% File information shown at the footer of the last page
\fileinfo{%
  \faCopyright{} 2020--2021, SUN Kai \hspace{0.5em}
  \creativecommons{by}{4.0} \hspace{0.5em}
  \githublink{Auto-SK}{Resume} \hspace{0.5em}
  Built with \faHeart{ and \LaTeXe} \hspace{0.5em}
  \faEdit{} \today
}

\name{凯}{孙}

% \tagline{\icon{\faBinoculars}} <position-to-look-for>}
% \tagline{<current-position>}

\photo{2.5cm}{pics/avatar.jpg}

\profile{
  \mobile{(+86) 18616366601}
  \email{sunkai.tech@gmail.com}
  \iconlink{\faLink}{https://sunkai.tech}{https://sunkai.tech} \\
  \university{上海科技大学}
  % \degree{电子科学与技术 \textbullet 硕士}
  \birthday{1996-03-23}
  \address{上海 \textbullet 浦东新区}
  \github{Auto-SK}
  % Custom information:
  % \icontext{<icon>}{<text>}
  % \iconlink{<icon>}{<link>}{<text>}
}

\begin{document}
\makeheader

%======================================================================
% Summary & Objectives
%======================================================================
{\onehalfspacing\hspace{2em}%
    研二,暑假可实习三个月,期望\textbf{软件开发}相关实习岗位。熟悉 Python、MATLAB,了解 C/C\#、SQL 等编程语言,具有良好的编程习惯及文档编写能力;会使用 Git、SSH、Docker 等开发工具;接触过 MySQL、SQL Server、Oracle 和 Redis 等数据库;熟练使用 Pycharm、Visual Studio 等集成开发环境;会基本的 Linux 操作,有 VPS 使用和个人网站部署经历;有很强的自主学习和解决问题能力,能够高效地进行团队沟通和协作。
    \par}

%======================================================================
\sectionTitle{教育背景}{\faGraduationCap}
%======================================================================
\begin{educations}
    \education%
    {2019.08}%
    {上海科技大学 (上海市与中科院共建) }%
    {信息科学与技术学院}%
    {电子科学与技术}%
    {硕士 (保送)}

    \separator{0.5ex}
    \education%
    {2015.08}%
    [2019.06]%
    {西安理工大学}%
    {自动化与信息工程学院}%
    {自动化}%
    {学士}
    [前 2\% (2/121) \textbullet 英语四级 (598) \textbullet 英语六级(443)]
    [西安理工大学大学生科技协会\textbf{副主席}\\
    西安理工大学工程训练中心创客工坊\textbf{社长}]
\end{educations}

%======================================================================
\sectionTitle{奖励荣誉}{\faAward}
%======================================================================
\begin{awards}
    \award
    {国家级}
    {
        \begin{itemize}
            \item \datedline{美国大学生交叉学科建模竞赛\textbf{二等奖}}{2018 年 04 月}
            \item \datedline{全国大学生电子设计竞赛\textbf{全国二等奖} (陕西省控制组第一名)}{2017 年 12 月}
            \item \datedline{全国大学生数学建模竞赛\textbf{全国二等奖} (前 3.4\%)}{2017 年 11 月}
            \item \datedline{全国大学生智能汽车竞赛\textbf{全国二等奖}}{2017 年 08 月}
            \item \datedline{国家励志奖学金 $\times$ 2}{2016 年 -- 2017 年}
        \end{itemize}
    }
    \separator{0.5ex}
    \award
    {校级}
    {
        \begin{itemize}
            \item \datedline{上海科技大学学业奖学金 $\times$ 2}{2019 年 -- 2020 年}
            \item \datedline{吴江创新奖学金}{2018 年 10 月}
            \item \datedline{创新成果一等奖}{2018 年 06 月}
        \end{itemize}
    }
\end{awards}

%======================================================================
\sectionTitle{项目经历}{\faListUl}
%======================================================================
\begin{experiences}
    \experience%
    [2019.09]
    {现在}%
    {\textbf{软件开发与部署 @ 中广核涡流检测数据仿真软件}}%
    [
    使用涡流原理和 FEM 对蒸汽发生管道和探头进行数值仿真,得到不同缺陷的图像和检出概率曲线。
        \begin{itemize}
            \item {编写被检对象和探头的 Mesh 程序和 FEM 计算程序,测试正确性并优化;}
            \item {编写探头阻抗特性数值计算程序,并使用 PyQt5 制作阻抗特性展示界面;}
            \item {使用 Berens 模型计算 POD,并将结果传输到前端界面;}
            \item {编写前后端接口文件协议,并与前端界面进行对接;}
            \item {调试程序并修改其中漏洞,发布程序并上线。}
        \end{itemize}
    ]
    [Python, FEM, PyQt, Numpy, Matplotlib, C\#, Git]
    \separator{0.5ex}

    \experience%
    [2020.12]
    {现在}%
    {\textbf{软件开发与部署 @ 金泽水面漂浮物智能检测系统}}%
    [
    训练部署 YOLOv5 模型,对金泽水库取水口和水文站水面漂浮物 (船、水葫芦) 进行实时检测和推送。
        \begin{itemize}
            \item {使用 OpenCV 读取 IP 摄像头的 RTSP 视频流,并解决因为带宽引起的丢帧问题;}
            \item {训练部署 YOLOv5 模型,并将其封装为 Detect 类;}
            \item {使用 Redis 作为缓存数据库,实时存储模型输出结果,并部署 Flask 服务器进行推送;}
            \item {将漂浮物信息保存到 Oracle 数据库,并将漂浮物图片保存到本地备份;}
            \item {采用多进程方式,将任务分为多个独立进程,进程间通过队列通信。}
        \end{itemize}]
    [Python, OpenCV, YOLOv5, Redis, Oracle, Flask, multiprocessing, Git]
\end{experiences}

\end{document}
