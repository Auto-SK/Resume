%======================================================================
\sectionTitle{项目经历}{\faListUl}
%======================================================================
\begin{experiences}
    \experience%
    [2019.09]
    {2021.07}%
    {\textbf{软件开发 @ 中广核涡流检测数据仿真软件 (国内首个核电站蒸汽发生管道电磁涡流检测仿真软件)}}%
    [
    使用涡流原理和 FEM 对蒸汽发生管道和探头进行数值仿真,获取不同缺陷的图像和检出概率曲线。
        \begin{itemize}
            \item {编写被检对象和探头的 Mesh 程序和 FEM 计算程序,测试正确性并优化;}
            \item {编写探头阻抗特性数值计算程序,并使用 PyQt5 制作阻抗特性计算界面;}
            \item {使用 Berens 模型计算 POD,并将结果传输到前端界面;}
            \item {基于 WinForm 框架制作界面,并编写相关的控件、布局、事件以及接口等程序;}
            \item {编写软件文档、接口协议、测试方案和测试样例,调试程序并修复漏洞;}
            \item {搭建并维护软件仓库,部署程序并最终成功发布。}
        \end{itemize}
    ]
    [Python, ECT, FEM, PyQt, Numpy, Matplotlib, C\#, WinForm, Git]
    \separator{0.5ex}

    \experience%
    [2020.12]
    {2021.05}%
    {\textbf{软件开发 @ 金泽水面漂浮物智能检测系统 (``十三五''水体污染控制与治理科技重大专项子项目)}}%
    [
    训练部署 YOLOv5 模型,对金泽水库取水口和水文站水面漂浮物 (船、水葫芦) 进行实时检测和推送。
        \begin{itemize}
            \item {使用 OpenCV 截取 IP 摄像头的 RTSP 视频流,并解决因为带宽引起的丢帧问题;}
            \item {训练部署 YOLOv5 模型,平均准确度在 0.98 以上,并将其封装为 Detect 类;}
            \item {将 YOLOv5 模型结果输入到 DeepSort 追踪器中,实现实时目标跟踪和计数;}
            \item {使用 Redis 做图片缓存,实时保存模型输出,并部署 Flask 服务器进行推送;}
            \item {保存漂浮物信息到 Oracle 数据库、部署 Nginx 图片服务器以供前端查询检测结果;}
            \item {采用多进程方式,将任务分为多个独立进程,进程间通过 Pipe 通信。}
        \end{itemize}]
    [Python, OpenCV, YOLOv5, DeepSort, Redis, Oracle, Flask, multiprocessing, Git]
\end{experiences}