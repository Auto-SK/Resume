%======================================================================
\sectionTitle{项目经历}{\faListUl}
%======================================================================
\begin{experiences}
    \experience%
    [2019.09]
    {现在}%
    {\textbf{项目负责人 @ 中广核涡流检测数据仿真软件}}%
    [
    使用涡流原理和 FEM 对蒸汽发生管道和探头进行数值仿真,得到不同缺陷的图像和检出概率曲线。
        \begin{itemize}
            \item {编写被检对象和探头的 Mesh 程序和 FEM 计算程序,测试正确性并优化;}
            \item {编写探头阻抗特性数值计算程序,并使用 PyQt5 制作阻抗特性计算界面;}
            \item {使用 Berens 模型计算 POD,并将结果传输到前端界面;}
            \item {编写前后端接口文件协议,并与前端界面进行对接;}
            \item {编写测试方案和测试样例,调试程序并修复漏洞。}
        \end{itemize}
    ]
    [Python, FEM, PyQt, Numpy, Matplotlib, C\#, Git]
    \separator{0.5ex}

    \experience%
    [2020.12]
    {2021.05}%
    {\textbf{项目负责人 @ 金泽水面漂浮物智能检测系统}}%
    [
    训练部署 YOLOv5 模型,对金泽水库取水口和水文站水面漂浮物 (船、水葫芦) 进行实时检测和推送。
        \begin{itemize}
            \item {使用 OpenCV 截取 IP 摄像头的 RTSP 视频流,并解决因为带宽引起的丢帧问题;}
            \item {训练部署 YOLOv5 模型,平均准确度在 0.98 以上,并将其封装为 Detect 类;}
            \item {将 YOLOv5 模型结果输入到 DeepSort 追踪器中,实现实时目标跟踪和计数;}
            \item {使用 Redis 做图片缓存,实时存储模型输出结果,并部署 Flask 服务器进行推送;}
            \item {将漂浮物信息保存到 Oracle 数据库,并将漂浮物图片保存到本地备份;}
            \item {采用多进程方式,将任务分为多个独立进程,进程间通过 Pipe 通信。}
        \end{itemize}]
    [Python, OpenCV, YOLOv5, DeepSort, Redis, Oracle, Flask, multiprocessing, Git]
\end{experiences}